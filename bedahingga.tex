\hypertarget{diferensiasi-numerik}{%
\chapter{Diferensiasi Numerik}\label{diferensiasi-numerik}}

    \begin{quote}
Diberikan fungsi \(f\), tentukan
\(\displaystyle{\frac{d^n f(x)}{dx^n}}\).
\end{quote}

    Di dunia nyata, seringkali menghitung turunan diperlukan dalam rentang
waktu yang singkat, meskipun harus mengorbankan akurasi. Ingat bahwa
turunan pertama mewakili tren dan turunan kedua mewakili kecepatan tren
berubah. Di kondisi yang lain, kita juga sering menemukan bahwa kita
membutuhkan turunan, namun poin data yang dipunyai tidak banyak. Hanya
titik per titik, sementara untuk melakukan \emph{interpolasi} maupun
\emph{curve fitting} memakan \emph{resource} yang lebih banyak.

    Ini mengakibatkan dibutuhkannya suatu konsep untuk menghitung fungsi
dengan cepat.

    Ingat bahwa dari definisi turunan, kita punya

\[ f'(x) = \lim \limits_{x \to 0} \frac{f(x+h) - f(x)}{h} \]

Jika kita mengambil nilai \(h\) tertentu, kita akan dapatkan \(f'(x)\)
mendekati nilai pada ruas kanan (tanpa limit).

    Sehingga kita bisa tulis ulang menjadi

\[ f'(x)  \approx \frac{f(x+h) - f(x)}{h}. \]

    Beda hingga atau \emph{finite difference} memiliki tiga metode.

\begin{itemize}
\tightlist
\item
  Beda maju
\item
  Beda mundur
\item
  Beda tengah
\end{itemize}

    \hypertarget{beda-hingga}{%
\section{Beda Hingga}\label{beda-hingga}}

    Dari persamaan sebelumnya \(f'(x) \approx \frac{f(x+h) - f(x)}{h}\),
kita bisa memilih nilai \(h>0\). Ini menjadi metode beda maju.

\[ f'(x) = \frac{f(x+h) - f(x)}{h}.\]

    Dari persamaan sebelumnya \(f'(x) \approx \frac{f(x+h) - f(x)}{h}\),
kita bisa memilih nilai \(h<0\). Ini menjadi metode beda maju. Jika kita
pilih \(k = -h\)

\[ f'(x) = \frac{f(x - k) - f(x)}{-k} = \frac{f(x) - f(x-k)}{k}\] atau
cukup ditulis \[ f'(x) = \frac{f(x) - f(x-k)}{k}.\]

    Kalau kita ganti simbol \(k\) dengan \(h\) didapat metode beda mundur.

\[ f'(x) = \frac{f(x) - f(x-h)}{h}.\]

    Untuk beda tengah agak berbeda. Kalau beda maju adalah melakukan
aproksimasi maju satu step (satu \(h\)) dan beda mundur melakukan satu
step ke belakang, kalau beda tengah ini melakukan aproksimasi dengan
maju setengah step dan mundur setengah step. Ini didapat dengan
menggabungkan beda maju dan beda mundur.

\[ f'(x) = \frac12 \left(  \frac{f(x+h) - f(x)}{h} +  \frac{f(x) - f(x-h)}{h} \right) = \frac{f(x+h) - f(x-h)}{2h} \]

    atau bila kita set \(2h \to h\), maka \(h \to \frac12 h\). Aproksimasi
menjadi

\[ f'(x) =   \frac{f(x + \frac12 h) - f(x- \frac 12 h)}{h} \]

